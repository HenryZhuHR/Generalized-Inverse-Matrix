%% An Introduction to LaTeX Thesis Template of Wuhan University
%%
%% Created by WHUTUG

% 用来设置附录中代码的样式



\documentclass[type=master,oneside]{whu-thesis}
\whusetup
{
  info               =
    {
      title          = {广义逆矩阵},
      title*         = {}, 
      student-number = {2021202120085},
      school         = {电子信息学院},
      author         = {朱鹤然},
      author*        = {Zhu Heran},
      subject        = {学科},
      major          = {电子信息学院},
      advisor        = {王文伟},
      direction      = {信息与通信工程},
      % date           = {2021/5},
      keywords       = {关键词 1 , 关键词 2 , 关键词 3 , 关键词 4 , 一个非常非常,非常非常长——的关键词 5},
      keywords*      = {key word 1 , key word 2 , key word 3 , key word 4 , {and a very very, very very long key word---the key word 5}},
    },
  style              =
    {
      graphics-path  = {{figures/}{data/}},
    },
  element            =
    {
      % abstract       = {pages/abstract},
      % bibliography   = {ref/refs},
      appendix       = {pages/appendix}
    }
}
% 
\usepackage{ctex}
\usepackage{listings}  % 排代码用的宏包
\usepackage{color}
\definecolor{codegreen}{rgb}{0,0.6,0}
\definecolor{codegray}{rgb}{0.5,0.5,0.5}
\definecolor{codepurple}{rgb}{0.58,0,0.82}
\definecolor{backcolour}{rgb}{0.95,0.95,0.92}
\usepackage{listings}
\usepackage{fontspec}
\setmonofont{Consolas}
\lstset{
    basicstyle        =   \ttfamily\footnotesize,
    keywordstyle      =   \color{codepurple}\bfseries,% 关键字颜色
    identifierstyle   =   \color{blue},
    commentstyle      =   \color{codegreen}% 注释颜色
    showstringspaces  =   false, % 不显示字符串内的空格
    showstringspaces  =   false, % 不显示字符串中的空格
    breaklines        =   true,   % 自动换行,建议不要写太长的行
    backgroundcolor   =   \color[rgb]{0.98,0.98,0.98},% 背景颜色
    numbers           =   left, % 显示行号 left right none
    xleftmargin       =   3em, %整体距左侧边线的距离为2em
    xrightmargin      =   1em,
}



\usepackage{hyperref}
\hypersetup{
	colorlinks=true,
	linkcolor=cyan,
	filecolor=blue,      
	urlcolor=red,
	citecolor=green,
}

\begin{document}
%%----------- 主体部分 ----------- %%
% \chapter{LU分解}
\section{LU 分解简介}
$LU$ 分解能够将 $m\times m$ 的满矩阵或稀疏矩阵 $A$ 分解成为一个 $m\times m$ 上三角矩阵 $U$ 和一个 $m\times m$ 经过置换的下三角矩阵 $L$,满足
\begin{equation}
    A = LU
    \label{LU-decomposition}
\end{equation}

另一种 $LU$ 分解的形式是将 $A$ 分解成为一个置换矩阵 $P$、 一个 $m\times m$ 上三角矩阵 $R$ 和一个 $m\times m$ 下单位三角矩阵 $Q$
\begin{equation}
    A = P^TLU
    \label{QRP-decomposition}
\end{equation}

LU 分解需要满足如下条件:
\begin{itemize}
    \item 矩阵是方阵
    \item 矩阵是可逆的,即矩阵是满秩矩阵,每一行都是独立向量
    \item 消元过程中没有0主元出现,也就是消元过程中不能出现行交换的初等变换。
\end{itemize}

\section{LU 分解的 Matlab 实现及结果验证}
用 Matlab 自带的函数 \lstinline|rand()| 创建一个大小为 $5\times 5$ 的随机矩阵


\begin{lstlisting}[language=Matlab]  
>> A=rand(5,5)*20
A =
    14.2243   8.4833    0.5844    4.7457    4.6319
    4.4349   10.1572   18.5771    9.1770    9.7780
    2.3484    1.7103   14.6066   19.2618   12.4812
    5.9335    5.2496    9.7722   10.9361   13.5827
    6.3756   16.0203   11.5705   10.4227    7.9103
\end{lstlisting}


向下取整后作为项目测试用的矩阵$A$
\begin{equation}
    A=\begin{bmatrix}
        14 & 8  & 0  & 4  & 4  \\
        4  & 10 & 18 & 9  & 9  \\
        2  & 1  & 14 & 19 & 12 \\
        5  & 5  & 9  & 10 & 13 \\
        6  & 16 & 11 & 10 & 7
    \end{bmatrix}
\end{equation}

随机产生列向量 $b$
\begin{lstlisting}[language=Matlab]  
>> b=rand(5,1)*20
b =
   12.9798
   16.0066
    9.0760
    8.6478
   16.5063
\end{lstlisting}

取 $b=[12,16,9,8,16]$

设置矩阵$A$
\begin{lstlisting}[language=Matlab]  
>> A = [
    14, 8,  0,  4,  4;
    4, 10, 18,  9,  9;
    2,  1, 14, 19, 12; 
    5,  5,  9, 10, 13; 
    6, 16, 11, 10,  7
    ]
A =
    14     8     0     4     4
     4    10    18     9     9
     2     1    14    19    12
     5     5     9    10    13
     6    16    11    10     7
\end{lstlisting}

在进行矩阵分解之前,我们需要验证该矩阵 $A$ 是否可逆
\begin{lstlisting}[language=Matlab]  
>> det(A)==0
ans =
  logical
   0
\end{lstlisting}
输出为逻辑 $0$ ,则表示矩阵 $A$ 的行列式不为 $0$,矩阵是可逆的,满足 $LU$ 分解的基本条件,可以进行分解


调用 Matlab 自带的函数 \lstinline|lu()| 进行矩阵分解,\lstinline|[L,U] = lu(A)| 函数可以将满秩矩阵 $A$ 分解为一个上三角矩阵 $U$ 和一个经过置换的下三角矩阵 $L$,使得
\begin{equation}
    A = LU
\end{equation}

\begin{lstlisting}[language=Matlab]  
>> [L,U] = lu(A)
L =

    1.0000         0         0         0         0
    0.2857    0.6136    0.7965    1.0000         0
    0.1429   -0.0114    1.0000         0         0
    0.3571    0.1705    0.5044    0.1823    1.0000
    0.4286    1.0000         0         0         0
U =
    14.0000    8.0000         0    4.0000    4.0000
          0   12.5714   11.0000    8.2857    5.2857
          0         0   14.1250   18.5227   11.4886
          0         0         0  -11.9799   -4.5366
          0         0         0         0    5.7024
\end{lstlisting}

为了验证结果,我们将上三角矩阵 $U$ 和下三角矩阵 $L$ 相乘
\begin{lstlisting}[language=Matlab]  
>> RES=L*U
RES =
   14.0000    8.0000         0    4.0000    4.0000
    4.0000   10.0000   18.0000    9.0000    9.0000
    2.0000    1.0000   14.0000   19.0000   12.0000
    5.0000    5.0000    9.0000   10.0000   13.0000
    6.0000   16.0000   11.0000   10.0000    7.0000
\end{lstlisting}
得到的结果矩阵 $B$ 与待分解矩阵一致


此外,该函数 \lstinline|[L,U,P] = lu(A)| 还可以返回一个置换矩阵 $P$,并满足 $A = P^TLU$。在此语法中,$L$ 是单位下三角矩阵,$U$ 是上三角矩阵。
\begin{lstlisting}[language=Matlab]  
>> [L,U,P] = lu(A)
L =
    1.0000         0         0         0         0
    0.4286    1.0000         0         0         0
    0.1429   -0.0114    1.0000         0         0
    0.2857    0.6136    0.7965    1.0000         0
    0.3571    0.1705    0.5044    0.1823    1.0000

U =
   14.0000    8.0000         0    4.0000    4.0000
         0   12.5714   11.0000    8.2857    5.2857
         0         0   14.1250   18.5227   11.4886
         0         0         0  -11.9799   -4.5366
         0         0         0         0    5.7024

P =
     1     0     0     0     0
     0     0     0     0     1
     0     0     1     0     0
     0     1     0     0     0
     0     0     0     1     0
\end{lstlisting}


验证分解的结果
\begin{lstlisting}[language=Matlab]  
>> RES=P'*L*U
RES =
   14.0000    8.0000         0    4.0000    4.0000
    4.0000   10.0000   18.0000    9.0000    9.0000
    2.0000    1.0000   14.0000   19.0000   12.0000
    5.0000    5.0000    9.0000   10.0000   13.0000
    6.0000   16.0000   11.0000   10.0000    7.0000
\end{lstlisting}


\section{利用 LU 分解求解线性方程组的 Matlab 实现}
假定需要求解的方程组为 $Ax=b$,即
\begin{equation}
    \begin{bmatrix}
        14 & 8  & 0  & 4  & 4  \\
        4  & 10 & 18 & 9  & 9  \\
        2  & 1  & 14 & 19 & 12 \\
        5  & 5  & 9  & 10 & 13 \\
        6  & 16 & 11 & 10 & 7
    \end{bmatrix}
    x = 
    \begin{bmatrix}
        12 \\ 16 \\ 9 \\ 8 \\ 16
    \end{bmatrix}
\end{equation}

求解上述方程组的过程如下
\begin{lstlisting}
>> b=[12;16;9;8;16];
>> [L,U,P] = lu(A);
>> y = L\(P*b)
y =
   12.0000
   10.8571
    7.4091
    0.0080
   -1.8752

>> x = U\y
x =
    0.7046
    0.3694
    0.6296
    0.1239
   -0.3288
\end{lstlisting}

验证结果
\begin{lstlisting}
>> RES=A*x
RES =
   12.0000
   16.0000
    9.0000
    8.0000
   16.0000
\end{lstlisting}


\section{LU 分解的 C++ 实现及结果验证}
\begin{lstlisting}[language=C++]
#include <iostream>
#include <Eigen/Dense>

using namespace std;
using namespace Eigen;
int main()
{
    Matrix<double, 5, 5> A;
    A<<14,8,0,4,4,4,10,18,9,9,2,1,14,19,12,5,5,9,10,13,6,16,11,10,7;
    cout << "matrix A:" << endl << A << endl << endl;

    FullPivLU<Matrix<double, 5, 5>> lu(A);

    Matrix<double, 5, 5> L = Matrix<double, 5, 5>::Identity();
    L.block<5,5>(0,0).triangularView<StrictlyLower>()=lu.matrixLU();
    cout << "matrix L:" << endl << L << endl << endl;

    Matrix<double, 5, 5> U = lu.matrixLU().triangularView<Upper>();
    cout << "matrix U:" << endl << U << endl << endl;

    cout << "reconstruct the original matrix A:" << endl;
    auto reconstruct = lu.permutationP().inverse() * L * U
                    * lu.permutationQ().inverse();
    cout << reconstruct << endl;
    cout << () << endl;

    return 0;
}
\end{lstlisting}

编译代码
\begin{lstlisting}[language=bash]
mkdir build
cd build
cmake -DCMAKE_BUILD_TYPE=Release ..
make -j8
cd ../bin
\end{lstlisting}
运行二进制文件,得到输出为
\begin{lstlisting}[language=bash]
E:\Projects\Matrix-Theory-Assignment\cpp\bin>lu.exe
matrix A:
14  8  0  4  4
4  10 18  9  9
2   1 14 19 12
5   5  9 10 13
6  16 11 10  7

matrix L:
           1           0           0           0           0
    0.526316           1           0           0           0
    0.210526    0.503401           1           0           0
    0.473684    0.615646 0.000613497           1           0
    0.526316    0.289116    0.226994    0.182325           1

matrix U:
       19        1        2       14       12
        0  15.4737  4.94737  3.63158 0.684211
        0        0  11.0884 -4.77551  1.12925
        0        0        0  9.13558  2.89387
        0        0        0        0  5.70244

reconstruct the original matrix A:
14  8  0  4  4
4  10 18  9  9
2   1 14 19 12
5   5  9 10 13
6  16 11 10  7
\end{lstlisting}
% \section{利用 LU 分解求解线性方程组的 C++ 实现}

% \chapter{QR 分解}
\section{QR 分解简介}
$QR$ 分解能够将 $m\times n (m\geq n)$ 的矩阵 $A$ 分解成为一个 $m\times n$ 上三角矩阵 $R$ 和一个 $m\times m$ 正交矩阵 $Q$

\begin{equation}
    A = QR
    \label{QR-decomposition}
\end{equation}

另一种 $QR$ 分解的形式是引入 $n\times n$ 置换矩阵 $P$ 使得 $AP$ 分解成为一个 $m\times n$ 上三角矩阵 $R$ 和一个 $m\times m$ 正交矩阵 $Q$
\begin{equation}
    AP = QR
    \label{QRP-decomposition}
\end{equation}

\section{QR 分解的 Matlab 实现及结果验证}
用 Matlab 自带的函数 \lstinline|rand()| 创建一个大小为 $3\times 5$ 的随机矩阵

\begin{lstlisting}[language=Matlab]  
>> A=rand(3,5)*20
A =
    16.2945   18.2675    5.5700   19.2978   19.1433
    18.1158   12.6472   10.9376    3.1523    9.7075
     2.5397    1.9508   19.1501   19.4119   16.0056
\end{lstlisting}


向下取整后作为项目测试用的矩阵$A$
\begin{equation}
    A=\begin{bmatrix}
        16&   18&    5&   19&   19\\
        18&   12&   10&    3&    9\\
        2&    1&   19&   19&   16
    \end{bmatrix}
\end{equation}

% 随机产生列向量 $b$
% \begin{lstlisting}[language=Matlab]  
% >> b=rand(5,1)*20
% b =
%     14.7572
%     5.3824
%     8.4567
% \end{lstlisting}

% 取 $b=[14,5,8]$




设置矩阵$A$
\begin{lstlisting}[language=Matlab]  
>> A = [16, 18, 5, 19, 19;18, 12, 10, 3, 9;2, 1, 19, 19, 16];
\end{lstlisting}

调用 Matlab 自带的函数 \lstinline|qr()| 进行矩阵分解,\lstinline|[Q,R] = qr(A)| 函数可以对 $3\times 5$ 矩阵 $A$ 进行 $QR$ 分解,满足式\ref{QR-decomposition}

\begin{lstlisting}[language=Matlab]  
>> [Q, R] = qr(A)
Q =
   -0.6621    0.7481    0.0449
   -0.7448   -0.6502   -0.1497
   -0.0828   -0.1325    0.9877
R =
  -24.1661  -20.9384  -12.3313  -16.3866  -20.6074
         0    5.5301   -5.2799    9.7449    6.2410
         0         0   17.4946   19.1707   15.3096
\end{lstlisting}

为了验证结果,我们将 $3\times 3$ 的正交矩阵 $Q$ 和 $3\times 5$ 的上三角矩阵 $R$ 相乘得到结果矩阵 $RES$
\begin{lstlisting}[language=Matlab]  
>> RES = Q * R
RES =
        16.0000   18.0000    5.0000   19.0000   19.0000
        18.0000   12.0000   10.0000    3.0000    9.0000
         2.0000    1.0000   19.0000   19.0000   16.0000
\end{lstlisting}
得到的结果矩阵 $RES$ 与待分解矩阵 $A$ 一致


此外,该函数 \lstinline|[Q,R,P] = qr(A)| 还会额外返回一个 $5\times 5$ 的置换矩阵 $P$,并满足 
\begin{equation}
    AP = QR
\end{equation}
\begin{lstlisting}[language=Matlab]  
>> [Q, R, P] = qr(A)
Q =
   -0.7027   -0.2969   -0.6465
   -0.1110   -0.8519    0.5118
   -0.7027    0.4314    0.5657
R =
  -27.0370  -14.6466  -17.9754  -14.6836  -25.5945
         0  -19.2218   -1.8064  -15.1357   -6.4056
         0         0   12.6342   -4.9298    1.3739
P =
     0     1     0     0     0
     0     0     0     1     0
     0     0     1     0     0
     1     0     0     0     0
     0     0     0     0     1
\end{lstlisting}

为了验证结果,我们将
$3\times 5$ 的待分解矩阵 $A$ 和 $5\times 5$ 的置换矩阵 $P$ 相乘得到结果矩阵 $RES_1$,并且将
$3\times 3$ 的正交矩阵 $Q$ 和 $3\times 5$ 的上三角矩阵 $R$ 相乘得到结果矩阵 $RES_2$
\begin{lstlisting}[language=Matlab]  
>> RES1 = A * P
RES2 = Q * R
RES1 =
    19    16     5    18    19
     3    18    10    12     9
    19     2    19     1    16
RES2 =
   19.0000   16.0000    5.0000   18.0000   19.0000
    3.0000   18.0000   10.0000   12.0000    9.0000
   19.0000    2.0000   19.0000    1.0000   16.0000
\end{lstlisting}
矩阵 $RES1$ 和 矩阵 $RES2$ 的结果一致


\section{QR 分解的 C++ 实现及结果验证}
\begin{lstlisting}[language=C++]  
#include <iostream>
#include <Eigen/Dense>

using namespace std;
using namespace Eigen;
int main()
{
    MatrixXf A(3, 5);
    A << 16, 18,  5, 19, 19, 
         18, 12, 10,  3,  9, 
          2,  1, 19, 19, 16;
    cout << "matrix A:" << endl << A << endl << endl;

    HouseholderQR<MatrixXf> qr(A);
    qr.compute(A);
    MatrixXf R = qr.matrixQR().triangularView<Upper>();
    MatrixXf Q = qr.householderQ();
    cout << "matrix Q:" << endl << Q << endl << endl;
    cout << "matrix R:" << endl << R << endl << endl;

    return 0;
}    
\end{lstlisting}

编译运行后
\begin{lstlisting}[language=bash]  
E:\Projects\Matrix-Theory-Assignment\cpp\bin>qr.exe
matrix A:
16 18  5 19 19
18 12 10  3  9
 2  1 19 19 16

matrix Q:
 -0.662085   0.748083  0.0448963
 -0.744845  -0.650238  -0.149654
-0.0827606  -0.132525   0.987719

matrix R:
-24.1661 -20.9384 -12.3313 -16.3866 -20.6074
       0  5.53012 -5.27993  9.74489  6.24104
       0        0  17.4946  19.1707  15.3096
\end{lstlisting}
该结果与 Matlab 运行结果一致
% \chapter{SVD 分解}
\section{SVD 分解简介}
$SVD$ (Singular Value Decomposition) 分解
usv
能够将 $m\times n$ 矩阵 $A$ 
分解成为 
$m\times m$ 的酉矩阵 $U$ 、 
$m\times n$ 的矩阵 $\Sigma$ 和 
$n\times n$ 的酉矩阵 $V$ ,满足
\begin{equation}
    A=U\Sigma V^T
    \label{SVD-decomposition}
\end{equation}

其中,酉矩阵 $U$ 和 $V$ 满足 $U^TU=I, V^TV=I$ ,$\Sigma$ 是一个对角矩阵,主对角线上的值就是奇异值


\section{SVD 分解的 Matlab 实现及结果验证}

用 Matlab 自带的函数 \lstinline|rand()| 创建一个大小为 $4\times 5$ 的随机矩阵

\begin{lstlisting}[language=Matlab]  
>> A=rand(4,5)*20
A =
    2.8377   19.1898   18.6799    7.8445    0.6367
    8.4352   13.1148   13.5747   13.1096    5.5385
    18.3147    0.7142   15.1548    3.4237    0.9234
    15.8441   16.9826   14.8626   14.1209    1.9426
\end{lstlisting}

向下取整后作为项目测试用的矩阵 +$A$
\begin{equation}
    A=\begin{bmatrix}
         2 & 19 & 18 &  7 &  0 \\
         8 & 13 & 13 & 13 &  5 \\
        18 &  0 & 15 &  3 &  0 \\
        15 & 16 & 14 & 14 &  1
    \end{bmatrix}
\end{equation}


设置矩阵$A$
\begin{lstlisting}[language=Matlab]  
>> A = [2, 19, 18, 7, 0; 8, 13, 13, 13, 5;
       18, 0, 15, 3, 0; 15, 16, 14, 14, 1]
\end{lstlisting}

调用 Matlab 自带的函数 \lstinline|svd()| 进行矩阵分解,\lstinline|[U, S, V] = svd(A)| 函数可以对 $4\times 5$ 矩阵 $A$ 进行 $SVD$ 分解,满足式\ref{SVD-decomposition}

\begin{lstlisting}[language=Matlab]  
>> [U, S, V] = svd(A)
U =
   -0.5101   -0.5302   -0.6739    0.0672
   -0.4906   -0.1470    0.4118   -0.7538
   -0.3720    0.8332   -0.3872   -0.1319
   -0.6006    0.0542    0.4758    0.6402
S =
   48.4506         0         0         0         0
         0   17.8100         0         0         0
         0         0    9.1462         0         0
         0         0         0    4.2053         0
V =
   -0.4262    0.7622    0.2312    0.3172    0.2886
   -0.5300   -0.6242    0.0177    0.4095    0.4018
   -0.6099    0.1012   -0.6477   -0.3814   -0.2300
   -0.4019   -0.1327    0.6708   -0.1810   -0.5815
   -0.0630   -0.0382    0.2771   -0.7440    0.6035
\end{lstlisting}

分解之后的奇异值为 $48.4506,17.8100,9.1462,4.2053$

首先验证 $U$、$V$ 是否是酉矩阵

\begin{lstlisting}[language=Matlab]  
>> RES=U*U'
RES =
    1.0000    0.0000    0.0000    0.0000
    0.0000    1.0000    0.0000         0
    0.0000    0.0000    1.0000    0.0000
    0.0000         0    0.0000    1.0000
\end{lstlisting}

\begin{lstlisting}[language=Matlab]  
>> RES=V*V'
RES =
    1.0000   -0.0000    0.0000    0.0000    0.0000
   -0.0000    1.0000    0.0000   -0.0000    0.0000
    0.0000    0.0000    1.0000    0.0000    0.0000
    0.0000   -0.0000    0.0000    1.0000   -0.0000
    0.0000    0.0000    0.0000   -0.0000    1.0000
\end{lstlisting}

求 $U^TU$ 和 $V^TV$ 之后,得到的结果均为单位阵 $I$ ,符合酉矩阵的定义。
验证分解的结果,求 $RES=USV^T$:
\begin{lstlisting}[language=Matlab]  
>> RES=U*S*V'
RES =
    2.0000   19.0000   18.0000    7.0000    0.0000
    8.0000   13.0000   13.0000   13.0000    5.0000
   18.0000    0.0000   15.0000    3.0000    0.0000
   15.0000   16.0000   14.0000   14.0000    1.0000
\end{lstlisting}




\section{SVD 分解的 Matlab 实现及应用}

对矩阵进行SVD分解之后,可以根据奇异值分解结果来确定矩阵的秩、列空间和零空间。



我们可以根据SVD分解的结果求解矩阵的秩(我们使用在上一部分中分解的数据进行)
\begin{lstlisting}[language=Matlab]  
>> s = diag(S)
rank_A = nnz(s)
s =
   48.4506
   17.8100
    9.1462
    4.2053
rank_A =
     4
\end{lstlisting}


我们可以根据 $SVD$ 分解的结果求解矩阵的列空间,使用 $U$ 中有对应的非零奇异值的列来计算 $A$ 的列空间的标准正交基。
\begin{lstlisting}[language=Matlab]  
>> column_basis = U(:,logical(s))
column_basis =
   -0.5101   -0.5302   -0.6739    0.0672
   -0.4906   -0.1470    0.4118   -0.7538
   -0.3720    0.8332   -0.3872   -0.1319
   -0.6006    0.0542    0.4758    0.6402
\end{lstlisting}

我们可以根据 $SVD$ 分解的结果求解矩阵的零空间,使用 $V$ 中有对应的零奇异值的列来计算 $A$ 的零空间的标准正交基。
\begin{lstlisting}[language=Matlab]  
>> null_basis = V(:,~s)

null_basis =

  空的 5×0 double 矩阵
\end{lstlisting}


\section{SVD 分解的 C++ 实现及结果验证}
\begin{lstlisting}[language=Matlab]  
#include <iostream>
#include <Eigen/Dense>

using namespace std;
using namespace Eigen;
int main()
{
    MatrixXf A(4, 5);
    A << 2 , 19 , 18 ,  7 ,  0 ,
        8 , 13 , 13 , 13 ,  5 ,
        18 ,  0 , 15 ,  3 ,  0 ,
        15 , 16 , 14 , 14 ,  1;
    cout << "matrix A:" << endl << A << endl << endl;


    JacobiSVD<MatrixXf> svd(A, ComputeThinU | ComputeThinV);
    cout << "Its singular values are:" << endl << svd.singularValues() << endl<< endl;
    cout << "matrix U: " << endl << svd.matrixU() << endl<< endl;
    cout << "matrix V: " << endl << svd.matrixV() << endl<< endl;
        
    return 0;
}
\end{lstlisting}


编译运行后
\begin{lstlisting}[language=Matlab]  
E:\Projects\Matrix-Theory-Assignment\cpp\bin>svd.exe
matrix A:
2  19 18  7  0
8  13 13 13  5
18  0 15  3  0
15 16 14 14  1

Its singular values are:
48.4506
  17.81
9.14625
4.20531

matrix U:
-0.510071 -0.530243 -0.673905 0.0672442
-0.490559 -0.147009  0.411754 -0.753789
-0.372042   0.83324 -0.387175 -0.131875
-0.600636 0.0542391  0.475822  0.640226

matrix V:
 -0.426226    0.76223   0.231176   0.317171
 -0.529999   -0.62425  0.0176852    0.40948
  -0.60986   0.101204  -0.647655  -0.381384
  -0.40191   -0.13272   0.670816  -0.180966
-0.0630215 -0.0382261   0.277118  -0.743994
\end{lstlisting}

得到的奇异值为 $48.4506, 17.81, 9.14625, 4.20531$ ,与Matlab中结果一致
\chapter{广义逆矩阵理论}
\section{Penrose 广义逆矩阵定义}
设矩阵 $A \in C^{m \times n}$ ,若矩阵 $X \in C^{n \times M}$满足如下四个 Penrose 方程
\begin{enumerate}
    \item $AXA=A$
    \item $XAX=X$
    \item $(AX)^H=AX$
    \item $(XA)^H=XA$
\end{enumerate}
则 $X$ 称为 $A$ 的\textbf{Moore-Penrose 逆},记为 $A^+$


\section{广义逆矩阵与线性方程的求解}
对于非齐次线性方程组
\begin{equation}
    Ax=b
\end{equation}
\end{document}