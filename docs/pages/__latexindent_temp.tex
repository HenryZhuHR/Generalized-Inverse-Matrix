\chapter{广义逆矩阵理论}
\section{Penrose 广义逆矩阵定义}
设矩阵 $A \in C^{m \times n}$ ,若矩阵 $X \in C^{n \times M}$满足如下四个 Penrose 方程
\begin{enumerate}
    \item $AXA=A$
    \item $XAX=X$
    \item $(AX)^H=AX$
    \item $(XA)^H=XA$
\end{enumerate}
则 $X$ 称为 $A$ 的\textbf{Moore-Penrose 逆},记为 $A^+$

Moore-Penrose 伪逆是一种矩阵,可在不存在逆矩阵的情况下作为逆矩阵的部分替代。此矩阵常被用于求解没有唯一解或有许多解的线性方程组。对于任何矩阵 $A$ 来说,伪逆 $B$ 都存在,是唯一的,并且具有与 $A^T $ 相同的维度。


\section{广义逆矩阵与线性方程组}
考虑非齐次线性方程组
\begin{equation}
    Ax=b
    \label{Eq.Linear_Eqtions}
\end{equation}
其中 $A \in C^{m \times n} ,b\in C^{m}$ 给定,而 $x\in C^n$ 为待定向量。如果存在向量 $x$ 使得方程组 \ref{Eq.Linear_Eqtions} 成立 ,那么称方程组\textbf{相容},否则称为\textbf{不相容}或\textbf{矛盾方程组}。

\subsection{线性方程组的相容条件极其解}
线性方程组 \ref{Eq.Linear_Eqtions} 相容的充要条件是
\begin{equation}
    AA^{(1)}b=b
\end{equation}
且其通解为
\begin{equation}
    x=A^{(1)}b+(I-A^{(1)}A)y
\end{equation}
其中 $y\in C^n$ 是任意的

相容方程组 \ref{Eq.Linear_Eqtions} 的通解同样可以表示特解 $x_0=A^{(1)}b$ 与通解 $z=(I-A^{(1)})y\in N(A)$ 之和:
\begin{equation}
    x=x_0+z
\end{equation}

对于相容方程组 \ref{Eq.Linear_Eqtions} 来说有无穷多个解,但是可以求出极小范数解
\begin{equation}
    \min\limits_{Ax=b}||x||
\end{equation}
极小范数解用欧式范数 $||\cdot||$ 计算

而相容方程组 \ref{Eq.Linear_Eqtions} 的极小范数解为 $x_0=A^{(1,4)}b$ ,其中$A^{(1,4)}b\in A\{1,4\}$

\section{广义逆矩阵与线性方程组的求解实验}

在 Matlab 中,提供了求 Moore-Penrose 伪逆的函数 \lstinline|pinv()| 

\subsection{实验1}
该部分 Matlab 仿真的矩阵 $A$ 来自课本习题 6.4 的第6题

\begin{equation}
    A=\begin{bmatrix}
        1, 0, 0, 1;
    1, 1, 0, 0;
    0, 1, 1, 0;
    0, 0, 1, 1
    \end{bmatrix}
\end{equation}


\begin{lstlisting}[language=Matlab]  
>> A=rand(5,5)*20
A =
    14.2243   8.4833    0.5844    4.7457    4.6319
    4.4349   10.1572   18.5771    9.1770    9.7780
    2.3484    1.7103   14.6066   19.2618   12.4812
    5.9335    5.2496    9.7722   10.9361   13.5827
    6.3756   16.0203   11.5705   10.4227    7.9103
\end{lstlisting}
