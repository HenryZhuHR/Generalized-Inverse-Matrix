\chapter{广义逆矩阵与线性方程组的求解实验}
在 Matlab 中,提供了求 Moore-Penrose 伪逆的函数 \lstinline|pinv()| 

\section{实验内容}
该部分 Matlab 仿真的矩阵 $A$ 来自课本习题 6.4 的第6题:

已知矩阵 $A=\begin{bmatrix}
    1 & 0 & 0 & 1 \\
    1 & 1 & 0 & 0 \\
    0 & 1 & 1 & 0 \\
    0 & 0 & 1 & 1
\end{bmatrix}$.
\begin{enumerate}
    \item 当 $b=(1,1,1,1)^T$ 时,方程组 $Ax=b$ 是否相容?
    \item 当 $b=(1,0,1,0)^T$ 时,方程组 $Ax=b$ 是否相容?
    \item 若方程组相容,求其通解和极小范数解;若方程组不相容,求其极小范数二乘解.
\end{enumerate}

\section{实验步骤}
\subsection{求矩阵的广义逆,并且验证矩阵可逆的条件}
\textbf{首先求出矩阵 $A$ 的逆}
\begin{lstlisting}[language=Matlab]  
>> A = [
    1, 0, 0, 1;
    1, 1, 0, 0;
    0, 1, 1, 0;
    0, 0, 1, 1
    ];
disp("Matrix A:")
disp(A)

B=pinv(A);

disp("Generalized Inverse Matrix of Matrix A:")
disp(B)
Matrix A:
    1     0     0     1
    1     1     0     0
    0     1     1     0
    0     0     1     1

Generalized Inverse Matrix of Matrix A:
     0.3750    0.3750   -0.1250   -0.1250
    -0.1250    0.3750    0.3750   -0.1250
    -0.1250   -0.1250    0.3750    0.3750
     0.3750   -0.1250   -0.1250    0.3750
\end{lstlisting}

对逆矩阵 $B$,我们需要\textbf{验证 Penrose 方程的四个条件\ref{Eq.Penrose_conditions}},以判断逆矩阵 $B$ 的类型

首先,验证 Penrose 方程 \ref{Eq.Penrose_conditions}-(1) 条件 $AXA = A$ 是否满足
\begin{lstlisting}[language=Matlab]
>> disp("Penrose condition 1:")
disp("AXA")
disp(A*B*A)
disp(A)

Penrose condition 1:
(1) AXA=A
    1.0000    0.0000    0.0000    1.0000
    1.0000    1.0000    0.0000    0.0000
   -0.0000    1.0000    1.0000    0.0000
   -0.0000         0    1.0000    1.0000

     1     0     0     1
     1     1     0     0
     0     1     1     0
     0     0     1     1
\end{lstlisting}
逆矩阵 $B$ 满足 Penrose方程 \ref{Eq.Penrose_conditions}-(1) 条件 $AXA = A$。因此,逆矩阵 $B$ 是 $A$ 的 \{1\}-逆。

验证 Penrose 方程 \ref{Eq.Penrose_conditions}-(2) 条件 $XAX = X$ 是否满足
\begin{lstlisting}[language=Matlab]
>> disp("Penrose condition 2:")
disp("(2) XAX=X")
disp(B*A*B)
disp(X)

Penrose condition 2:
(2) XAX=X
    0.3750    0.3750   -0.1250   -0.1250
   -0.1250    0.3750    0.3750   -0.1250
   -0.1250   -0.1250    0.3750    0.3750
    0.3750   -0.1250   -0.1250    0.3750

    0.3750    0.3750   -0.1250   -0.1250
   -0.1250    0.3750    0.3750   -0.1250
   -0.1250   -0.1250    0.3750    0.3750
    0.3750   -0.1250   -0.1250    0.3750
\end{lstlisting}
逆矩阵 $B$ 满足 Penrose方程 \ref{Eq.Penrose_conditions}-(2) 条件 $XAX = X$。因此,逆矩阵 $B$ 是 $A$ 的 \{1,2\}-逆。

验证 Penrose 方程 \ref{Eq.Penrose_conditions}-(3) 条件 $(AX)^H = AX$ 是否满足
\begin{lstlisting}[language=Matlab]
>> disp("Penrose condition 3:")
disp("(3) (AX)^H=AX")
disp((A*B)')
disp(A*B)

Penrose condition 3:
(3) (AX)^H=AX
    0.7500    0.2500   -0.2500    0.2500
    0.2500    0.7500    0.2500   -0.2500
   -0.2500    0.2500    0.7500    0.2500
    0.2500   -0.2500    0.2500    0.7500

    0.7500    0.2500   -0.2500    0.2500
    0.2500    0.7500    0.2500   -0.2500
   -0.2500    0.2500    0.7500    0.2500
    0.2500   -0.2500    0.2500    0.7500
\end{lstlisting}
逆矩阵 $B$ 满足 Penrose方程 \ref{Eq.Penrose_conditions}-(3) 条件 $(AX)^H = AX$。因此,逆矩阵 $B$ 是 $A$ 的 \{1,2,3\}-逆。

验证 Penrose 方程 \ref{Eq.Penrose_conditions}-(4) 条件 $(XA)^H = XA$ 是否满足
\begin{lstlisting}[language=Matlab]
>> disp("Penrose condition 4:")
disp("(4) (XA)^H=XA")
disp((B*A)')
disp(B*A)

Penrose condition 4:
(4) (XA)^H=XA
    0.7500    0.2500   -0.2500    0.2500
    0.2500    0.7500    0.2500   -0.2500
   -0.2500    0.2500    0.7500    0.2500
    0.2500   -0.2500    0.2500    0.7500

    0.7500    0.2500   -0.2500    0.2500
    0.2500    0.7500    0.2500   -0.2500
   -0.2500    0.2500    0.7500    0.2500
    0.2500   -0.2500    0.2500    0.7500
\end{lstlisting}
逆矩阵 $B$ 满足 Penrose方程 \ref{Eq.Penrose_conditions}-(4) 条件 $(XA)^H = XA$。因此,逆矩阵 $B$ 是完全满足 Penrose 方程的四个条件\ref{Eq.Penrose_conditions},所以逆矩阵 $B$ 是$A$ 的 Moore-Penrose 逆,即 $B=A^+$。

\subsection{利用广义逆矩阵求解相容方程组的极小范数解}
\textbf{第 1 问}:

当 $b=(1,1,1,1)^T$ 时,根据方程组 $Ax=b$ 相容的条件 (式\ref{Eq.LinearEqtions_ConsistentCondition}),我们通过计算矩阵的逆 $B=A^{(1)}$ 是否满足 $Ax_0=b$ 和 $ABb=b$ 的条件以判断方程组 $Ax=b$ 是否相容。

\begin{lstlisting}[language=Matlab]  
>> b=[1;1;1;1];
>> x_0=B*b;
disp("x_0=")
disp(x_0)
disp("compute A*x_0=")
disp(A*x_0)
disp("compute A*A^+*b=")
disp(A*B*b)

x_0=
    0.5000
    0.5000
    0.5000
    0.5000

compute A*x_0=
    1.0000
    1.0000
    1.0000
    1.0000

compute A*A^+*b=
    1.0000
    1.0000
    1.0000
    1.0000
\end{lstlisting}

从计算结果可以看出,对于 $A$ 的广义逆 $B=A^{(1)}$ 满足 $Ax_0=b$ 和 $ABb=b$ 条件。因此,对于 $b=(1,1,1,1)^T$ 时的方程组 $Ax=b$ 是相容的。

由于前面我们已经验证,逆矩阵 $B$ 是 $A$ 的 Moore-Penrose 逆,所以可以得出当 $b=(1,1,1,1)^T$ 时方程组 $Ax=b$ 的方程组的极小范数解为 
\begin{equation}
    x = A^+b = \frac{1}{2}
    \begin{bmatrix}
        1 \\ 1 \\ 1 \\ 1
    \end{bmatrix}
\end{equation}


通解为
\begin{equation}
    \begin{aligned}
        x   &= A^{(1)}b + (I-A^{(1)}A)y \\
            &= \frac{1}{2}
            \begin{bmatrix}
                1 \\ 1 \\ 1 \\ 1
            \end{bmatrix} +
            \frac{1}{4}
            \begin{bmatrix}
                1  & -1 & 1  & -1 \\
                -1 & 1  & -1 & 1  \\
                1  & -1 & 1  & -1 \\
                -1 & 1  & -1 & 1
            \end{bmatrix}
            \begin{bmatrix}
                \xi_1 \\ \xi_2 \\ \xi_3 \\ \xi_4
            \end{bmatrix}
    \end{aligned}
\end{equation}

对于相容方程组 \ref{Eq.Linear_Eqtions} 来说有无穷多个解,但是可以求出极小范数解,极小范数解用欧式范数 $||\cdot||$ 计算
\begin{equation}
    \min\limits_{Ax=b}||x||
\end{equation}

在 Matlab 中,我们也可以用函数 \lstinline|norm()|  验证该范数的值
\begin{lstlisting}[language=Matlab] 
>> disp("\min\limits_x ||x||")
disp(norm(x_0))

\min\limits_x ||x||
    1.0000
\end{lstlisting}


\subsection{利用广义逆矩阵求解矛盾方程组极小范数最小二乘解}

\textbf{第 2 问}:

当 $b=(1,0,1,0)^T$ 时,根据方程组 $Ax=b$ 相容的条件 (式\ref{Eq.LinearEqtions_ConsistentCondition}) ,我们通过计算矩阵的逆 $B=A^{(1)}$ 是否满足 $Ax_0=b$ 和 $ABb=b$ 的条件以判断方程组 $Ax=b$ 是否相容。

\begin{lstlisting}[language=Matlab]  
>> b=[1;0;1;0];
>> x_0=B*b;
disp("x_0=")
disp(x_0)
disp("compute A*x_0=")
disp(A*x_0)
disp("compute A*A^+*b=")
disp(A*B*b)

x_0=
    0.2500
    0.2500
    0.2500
    0.2500

compute A*x_0=
    0.5000
    0.5000
    0.5000
    0.5000

compute A*A^+*b=
    0.5000
    0.5000
    0.5000
    0.5000
\end{lstlisting}

从计算结果可以看出,对于 $A$ 的广义逆 $B=A^{(1)}$ 并不满足 $Ax_0=b$ 和 $ABb=b$ 条件。因此,对于 $b=(1,0,1,0)^T$ 时的方程组 $Ax=b$ 是不相容的,是矛盾方程组。

由此可以得出当 $b=(1,0,1,0)^T$ 时,矛盾方程组 $Ax=b$ 的极小范数最小二乘解是唯一的,并且可以由 Moore-Penrose 逆表示,由于前面我们已经验证,逆矩阵 $B$ 是 Moore-Penrose 逆,所以方程组的极小范数最小二乘解为
\begin{equation}
    x=A^+b=\frac{1}{4}\begin{bmatrix}
        1 \\ 1 \\ 1 \\ 1
    \end{bmatrix}
\end{equation}
\begin{lstlisting}[language=Matlab]  
>> b=[1;0;1;0];
>> x_0=B*b;
disp("x_0=")
disp(x_0)

x_0=
    0.2500
    0.2500
    0.2500
    0.2500
\end{lstlisting}


对于矛盾方程组 \ref{Eq.Linear_Eqtions} 来说有无穷多个解,但是可以求出极小范数解,极小范数解用欧式范数 $||\cdot||$ 计算
\begin{equation}
    \min\limits_{Ax=b}||x||
\end{equation}

在 Matlab 中,我们也可以用函数 \lstinline|norm()|  验证该范数的值
\begin{lstlisting}[language=Matlab] 
>> disp("\min\limits_x ||Ax-b||")
disp(norm(A*x_0-b))

\min\limits_x ||Ax-b||
    1.0000
\end{lstlisting}