\chapter{SVD 分解}
\section{SVD 分解简介}
$SVD$ (Singular Value Decomposition) 分解
usv
能够将 $m\times n$ 矩阵 $A$ 
分解成为 
$m\times m$ 的酉矩阵 $U$ 、 
$m\times n$ 的矩阵 $\Sigma$ 和 
$n\times n$ 的酉矩阵 $V$ ,满足
\begin{equation}
    A=U\Sigma V^T
    \label{SVD-decomposition}
\end{equation}

其中,酉矩阵 $U$ 和 $V$ 满足 $U^TU=I, V^TV=I$ ,$\Sigma$ 是一个对角矩阵,主对角线上的值就是奇异值


\section{SVD 分解的 Matlab 实现及结果验证}

用 Matlab 自带的函数 \lstinline|rand()| 创建一个大小为 $4\times 5$ 的随机矩阵

\begin{lstlisting}[language=Matlab]  
>> A=rand(4,5)*20
A =
    2.8377   19.1898   18.6799    7.8445    0.6367
    8.4352   13.1148   13.5747   13.1096    5.5385
    18.3147    0.7142   15.1548    3.4237    0.9234
    15.8441   16.9826   14.8626   14.1209    1.9426
\end{lstlisting}

向下取整后作为项目测试用的矩阵 +$A$
\begin{equation}
    A=\begin{bmatrix}
         2 & 19 & 18 &  7 &  0 \\
         8 & 13 & 13 & 13 &  5 \\
        18 &  0 & 15 &  3 &  0 \\
        15 & 16 & 14 & 14 &  1
    \end{bmatrix}
\end{equation}


设置矩阵$A$
\begin{lstlisting}[language=Matlab]  
>> A = [2, 19, 18, 7, 0; 8, 13, 13, 13, 5;
       18, 0, 15, 3, 0; 15, 16, 14, 14, 1]
\end{lstlisting}

调用 Matlab 自带的函数 \lstinline|svd()| 进行矩阵分解,\lstinline|[U, S, V] = svd(A)| 函数可以对 $4\times 5$ 矩阵 $A$ 进行 $SVD$ 分解,满足式\ref{SVD-decomposition}

\begin{lstlisting}[language=Matlab]  
>> [U, S, V] = svd(A)
U =
   -0.5101   -0.5302   -0.6739    0.0672
   -0.4906   -0.1470    0.4118   -0.7538
   -0.3720    0.8332   -0.3872   -0.1319
   -0.6006    0.0542    0.4758    0.6402
S =
   48.4506         0         0         0         0
         0   17.8100         0         0         0
         0         0    9.1462         0         0
         0         0         0    4.2053         0
V =
   -0.4262    0.7622    0.2312    0.3172    0.2886
   -0.5300   -0.6242    0.0177    0.4095    0.4018
   -0.6099    0.1012   -0.6477   -0.3814   -0.2300
   -0.4019   -0.1327    0.6708   -0.1810   -0.5815
   -0.0630   -0.0382    0.2771   -0.7440    0.6035
\end{lstlisting}

分解之后的奇异值为 $48.4506,17.8100,9.1462,4.2053$

首先验证 $U$、$V$ 是否是酉矩阵

\begin{lstlisting}[language=Matlab]  
>> RES=U*U'
RES =
    1.0000    0.0000    0.0000    0.0000
    0.0000    1.0000    0.0000         0
    0.0000    0.0000    1.0000    0.0000
    0.0000         0    0.0000    1.0000
\end{lstlisting}

\begin{lstlisting}[language=Matlab]  
>> RES=V*V'
RES =
    1.0000   -0.0000    0.0000    0.0000    0.0000
   -0.0000    1.0000    0.0000   -0.0000    0.0000
    0.0000    0.0000    1.0000    0.0000    0.0000
    0.0000   -0.0000    0.0000    1.0000   -0.0000
    0.0000    0.0000    0.0000   -0.0000    1.0000
\end{lstlisting}

求 $U^TU$ 和 $V^TV$ 之后,得到的结果均为单位阵 $I$ ,符合酉矩阵的定义。
验证分解的结果,求 $RES=USV^T$:
\begin{lstlisting}[language=Matlab]  
>> RES=U*S*V'
RES =
    2.0000   19.0000   18.0000    7.0000    0.0000
    8.0000   13.0000   13.0000   13.0000    5.0000
   18.0000    0.0000   15.0000    3.0000    0.0000
   15.0000   16.0000   14.0000   14.0000    1.0000
\end{lstlisting}




\section{SVD 分解的 Matlab 实现及应用}

对矩阵进行SVD分解之后,可以根据奇异值分解结果来确定矩阵的秩、列空间和零空间。



我们可以根据SVD分解的结果求解矩阵的秩(我们使用在上一部分中分解的数据进行)
\begin{lstlisting}[language=Matlab]  
>> s = diag(S)
rank_A = nnz(s)
s =
   48.4506
   17.8100
    9.1462
    4.2053
rank_A =
     4
\end{lstlisting}


我们可以根据 $SVD$ 分解的结果求解矩阵的列空间,使用 $U$ 中有对应的非零奇异值的列来计算 $A$ 的列空间的标准正交基。
\begin{lstlisting}[language=Matlab]  
>> column_basis = U(:,logical(s))
column_basis =
   -0.5101   -0.5302   -0.6739    0.0672
   -0.4906   -0.1470    0.4118   -0.7538
   -0.3720    0.8332   -0.3872   -0.1319
   -0.6006    0.0542    0.4758    0.6402
\end{lstlisting}

我们可以根据 $SVD$ 分解的结果求解矩阵的零空间,使用 $V$ 中有对应的零奇异值的列来计算 $A$ 的零空间的标准正交基。
\begin{lstlisting}[language=Matlab]  
>> null_basis = V(:,~s)

null_basis =

  空的 5×0 double 矩阵
\end{lstlisting}


\section{SVD 分解的 C++ 实现及结果验证}
\begin{lstlisting}[language=Matlab]  
#include <iostream>
#include <Eigen/Dense>

using namespace std;
using namespace Eigen;
int main()
{
    MatrixXf A(4, 5);
    A << 2 , 19 , 18 ,  7 ,  0 ,
        8 , 13 , 13 , 13 ,  5 ,
        18 ,  0 , 15 ,  3 ,  0 ,
        15 , 16 , 14 , 14 ,  1;
    cout << "matrix A:" << endl << A << endl << endl;


    JacobiSVD<MatrixXf> svd(A, ComputeThinU | ComputeThinV);
    cout << "Its singular values are:" << endl << svd.singularValues() << endl<< endl;
    cout << "matrix U: " << endl << svd.matrixU() << endl<< endl;
    cout << "matrix V: " << endl << svd.matrixV() << endl<< endl;
        
    return 0;
}
\end{lstlisting}


编译运行后
\begin{lstlisting}[language=Matlab]  
E:\Projects\Matrix-Theory-Assignment\cpp\bin>svd.exe
matrix A:
2  19 18  7  0
8  13 13 13  5
18  0 15  3  0
15 16 14 14  1

Its singular values are:
48.4506
  17.81
9.14625
4.20531

matrix U:
-0.510071 -0.530243 -0.673905 0.0672442
-0.490559 -0.147009  0.411754 -0.753789
-0.372042   0.83324 -0.387175 -0.131875
-0.600636 0.0542391  0.475822  0.640226

matrix V:
 -0.426226    0.76223   0.231176   0.317171
 -0.529999   -0.62425  0.0176852    0.40948
  -0.60986   0.101204  -0.647655  -0.381384
  -0.40191   -0.13272   0.670816  -0.180966
-0.0630215 -0.0382261   0.277118  -0.743994
\end{lstlisting}

得到的奇异值为 $48.4506, 17.81, 9.14625, 4.20531$ ,与Matlab中结果一致